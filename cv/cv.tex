% Gabriel Ryan's Resume

%\documentclass[9pt,letterpaper]{article} % Font size and paper type
\documentclass{article} % Font size and paper type

\usepackage[parfill]{parskip} % Remove paragraph indentation
\usepackage{array} % Required for boldface (\bf and \bfseries) tabular columns
\usepackage[hidelinks]{hyperref} % Required for boldface (\bf and \bfseries) tabular columns
%\usepackage{ifthen} % Required for ifthenelse statements
\usepackage{geometry}
\usepackage{color}
\usepackage[usenames,dvipsnames]{xcolor}
\usepackage{graphicx}
\usepackage{verbatim} %multiline comments
\usepackage{url}
\usepackage{enumitem}

\definecolor{lightgrey}{gray}{.95}


%\usepackage{fontspec}
\usepackage[scale=0.90]{noto-mono}

%\usepackage[scaled]{berasans}
%\renewcommand*\familydefault{\sfdefault}  %% Only if the base font of the document is to be sans serif

%\usepackage[usefilenames,RMstyle=Light,SSstyle=Light,TTstyle=Light,DefaultFeatures={Ligatures=Common}]{plex-otf} %
%\renewcommand*\sfdefault{lmssq}
%\usepackage[sfdefault]{roboto}  %% Option 'sfdefault' only if the base font of the document is to be sans serif
\renewcommand*\familydefault{\sfdefault} %% Only if the base font of the document is to be sans serif
%\usepackage[T1]{fontenc}
%\defaultfontfeatures{Mapping=tex-text,Scale=MatchLowercase}

%\usepackage{helvet}
%\usepackage{lmodern}
%\usepackage{tgtermes}
%\usepackage{palatino}


\geometry{right=1in,left=.8in,top=.6in,bottom=.7in}
\pagestyle{empty} % Suppress page numbers

\newcommand{\tbf}[1]{\textbf{#1}}

%%%%%%%%%%%%%%%%%%%%%%%%% Commands %%%%%%%%%%%%%%%%%%%%%%%%%
%\rsection{<section title>}
\newcommand{\rsection}[1]{
  \hspace{-0.4cm}\vspace{0.1cm}
\colorbox{lightgrey}{
\begin{minipage}{1.07\linewidth}
\vspace{0.22cm}
\fontsize{14pt}{16pt}\selectfont #1
%\rule[9pt]{\linewidth}{1.2pt}
\vspace{0.12cm}
\end{minipage}
}
\vspace*{-0.1cm}
}
%\rjob{<job title>}{<time>}
\newcommand{\rjob}[2]{
  \hspace*{-0.3cm}
{\fontsize{10pt}{12pt}\selectfont #1} \hfill #2
\vspace*{0.1cm}
\hspace*{-1.2cm}
}
%\begin{ritemize}...\end{ritemize}
\newenvironment{ritemize}{
\hspace*{-0.8cm}
\begin{minipage}{1.05\linewidth}
\begin{itemize}
}{
\end{itemize}
\end{minipage}
%\vspace{-0.1cm}
}
%\ritem
\newcommand{\ritem}{
\item[-]
%\hspace{-8pt}
}


\begin{document}
%%%%%%%%%%%%%%%%%%%%%%%%% Header %%%%%%%%%%%%%%%%%%%%%%%%%
%\begin{center}
\hspace*{-0.45cm}
{\fontsize{22pt}{22pt}\selectfont Gabriel Ryan}\\
%\vspace{1.2cm}
%\vspace*{-0.5cm}
%\hspace*{0.1cm}
\begin{minipage}{\linewidth}
\vspace{0.1cm}
  {\fontsize{12}{12}\selectfont
  %{\normalsize
  %\texttt{
      \href{mailto:gabe@cs.columbia.edu}{gabe@cs.columbia.edu} | \url{www.cs.columbia.edu/~gabe}
    %}
  }
%3260 Henry Hudson Pkwy. Bronx, NY 10461
%116th and Broadway New York, NY 10027
%}
\end{minipage}
\vspace{-0.15cm}
%\end{center}

\rsection{Research Focus:}

My research interests are at the intersection of Machine Learning, Programming Languages, and Software Engineering/Security. My recent work focuses on using LLMs and probabilistic methods for program testing and repair.


%%%%%%%%%%%%%%%%%%%%%%%%% Education %%%%%%%%%%%%%%%%%%%%%%%%%
\rsection{Education:}

\rjob{\textbf{Ph.D. Columbia University}}{Anticipated December 2023}\\
\begin{ritemize}
\ritem Ph.D. Candidate in Computer Science advised by Professor Suman Jana.
%\ritem M.S. in Computer Science (GPA 3.97)
    %\ritem Anticipated graduation \textbf{December 2023}.
    %\ritem Awarded NDSEG Fellowship for proposal "Proximal Gradient Analysis for Vulnerability Detection and Defense."
%\ritem Received NSF Graduate Research Fellowship Honorable Mention for proposal "Modeling and Simulating Adversarial User Behavior with Sequential VAEs."
\end{ritemize}

\rjob{\textbf{M.S. Columbia University}}{December 2017}\\
\begin{ritemize}
  \ritem M.S. in Computer Science %(GPA 3.97)
  \ritem Conducted research in modeling and simulating user behavior for Intrusion Detection Lab.
  \ritem Selected Coursework: Security, Deep Learning, Adv. Deep Learning, Data Analytics, Data Exploration Systems, Machine Learning, Natural Language Processing, Operating Systems, Compilers, High Perf. Computing
\end{ritemize}

\rjob{\textbf{B.S. Swarthmore College}}{June 2013}\\
\begin{ritemize}
    \ritem B.S. Engineering, B.A. Computer Science %(In Major GPA 3.83)
%\ritem GPA: \begin{tabular}[t]{lr} Computer Science & 3.87\\
%Engineering & 3.46 \\
%Overall & 3.56
%\end{tabular}
%\ritem Spring 2012 Semester Abroad at University of Auckland in New Zealand.
\ritem Selected Coursework: Adv. Algorithms, Theory of Computation, Artificial Intelligence, Probabilistic Robotics, Computer Systems, Computer Languages, Applied Statistics, Optimization, Linear Systems Analysis
%\ritem Coursework: Data Structures and Algorithms, Algorithms (Advanced), Theory of Computation, Artificial Intelligence, Optimization in Operations Research, Linear System Modeling.
\end{ritemize}
\vspace{0.25cm}



%%%%%%%%%%%%%%%%%%%%%%%%% Experience %%%%%%%%%%%%%%%%%%%%%%%%%
\rsection{Research Experience:}


\rjob{\textbf{Doctoral Researcher}, Columbia Security Lab}{Sep 2018 - Present}\\
\begin{ritemize}
    %\ritem Working with advisor Prof. Suman Jana using deep learning to develop new approaches to program analysis.
    \ritem Developing new approaches to linux kernel syzbot crash report explanation \& bug repair with LLMs.
    \ritem Developed novel linux kernel race prediction analysis with probabilistic and spectral learning. Published in Oakland S\&P 2023.
    \ritem Developed \textit{Proximal Gradient Analysis} (PGA), a new method for dataflow analysis that uses nonsmooth optimization methods. Implemented in LLVM. Published in USENIX Security 2021.
    \ritem Designed the \textit{Continuous Logic Network} (CLN), a neural architecture for logical learning and reasoning, and applied it to learning program invariants for verification. Implemented library for constructing CLNs in Pytorch. Published in ICLR 2020 and PLDI 2020.
\end{ritemize}


\rjob{\textbf{Applied Science Intern}, AWS A.I. Labs}{Jun 2023 - Aug 2023}\\
\begin{ritemize}
  \ritem Research internship with AWS CodeWhisperer research group advised by visiting Prof. Baishahki Ray.
  \ritem Developed novel approach to regression testing with LLMs using static analysis to prompt the model to reason symbolically about program execution paths.
  \ritem Achieved improvements of $2\times$ coverage and $3\times$ correct test generations over baseline LLM regression test generation approaches when evaluated on CodeGen2 and OpenAI GPT-3.5 and GPT-4 models.
\end{ritemize}


\rjob{\textbf{Research Intern}, Microsoft Research}{Jun 2021 - Aug 2021}\\
\begin{ritemize}
    \ritem Research internship with MSR RiSE group advised by Sr. Pr. Scientists Todd Mytkowitz and Shuvendu Lahiri.
    \ritem Developed \textit{TOGA: A Neural Method for Test Oracle Generation} using Transformers (CodeBERT) and a specialized grammar to generate unit tests and demonstrated a 170\% improvement over prior work in bug detection.
    \ritem Published work in ICSE 2022, awarded ACM SigSoft Distinguished Paper Award.
\end{ritemize}


\rjob{\textbf{MS Graduate Research Assistant}, Columbia IDS Lab}{Fall 2016 - Fall 2017}\\
\begin{ritemize}
  \ritem Developed novel method for modeling and generating user behavior data with neural networks advised by Prof. Salvatore Stolfo.
  \ritem Designed architecture combining recurrent neural networks with multi task prediction to jointly model events and times and a real-time sequence and demonstrated generated event sequences effectively simulated human user behavior. Published in IEEE S\&P Workshops 2018.
  %\ritem Developed framework for Simulated User Bots on mobile devices using Selenium and Appium. Designed Python API to enable automation of common user tasks, such as browsing the web and sending emails.
\end{ritemize}

%\rjob{\textbf{Seminar Research}, Data Exploration Systems}{Spring 2017 - Spring 2018}\\
%\begin{ritemize}
    %\ritem Working in Prof. Eugene Wu's lab, developed \emph{Pixel Approximate Entropy} (PAE), a new method for measuring visual complexity in charts using approximate entropy.  Developed and conducted automated Amazon Mechanical Turk user studies demonstrating that PAE effectively predicted user's ability to perform accurately perform visual analysis tasks. Published in IEEE Infoviz 2018.
%\end{ritemize}

%\rjob{\textbf{Seminar Research}, Adv Big Data Analytics}{Spring 2017}\\
%\begin{ritemize}
  %\ritem Developed graph data queries and synethetic graph data in Tensorflow to demonstrate feasibility of using a deep learning approach to learn graph query parameters.
  %\ritem Implemented Pagerank Algorithm in Tensorflow and demonstrated dampening parameter could be learned from labeled graph data using back propogation.
%\end{ritemize}

%\rjob{\textbf{Research Fellow}, Bryn Mawr Intelligent Systems Lab}{Summer 2013}\\
%\begin{ritemize}
%\ritem Developed tools using ROSJAVA package to run Java reinforcement learning algorithms on ROS robots and simulators. Implemented the HORDE reinforcement learning algorithm and demonstrated multitask learning outperforms HORDE for robotic sensor prediction.
%\end{ritemize}

%\pagebreak{}

\rjob{\textbf{Research Intern}, Woods Hole Oceanographic Institute}{January 2012}\\
\begin{ritemize}
%\ritem Automated filtering and parsing of data from tow tank tests of a sub-Arctic current meter in MATLAB, setting research weeks ahead of schedule.
\ritem Conducted frequency analysis of velocity data and developed burst sampling system that improved the signal to noise ratio by a factor of 3 while minimizing computation and radio usage.
\ritem Published work in proceedings of Oceans 2012 conference.
\end{ritemize}
%\rjob{\textbf{Research Fellow}, Swarthmore College Engineering}{Summer 2011}\\
%\begin{ritemize}
%%\ritem Awarded Surdna Summer Research Fellowship to design a wireless thermostat system.
%\ritem Developed a wireless thermostat prototype around an MSP430 microcontroller and Zigbee radios that used low power event driven processing to achieve an estimated life expectancy of 5+ years on a coin cell battery.
%\ritem Wrote MATLAB Zigbee library to interface with Zigbee radios, allowing user to configure and use Zigbee radios through a simplified API.
%\end{ritemize}
%\pagebreak{}

\vspace{0.2cm}
\rsection{Professional Experience:}

\rjob{\textbf{Software Engineer}, Allure Security Technology}{Sep 2015 - August 2018}\\
\begin{ritemize}
\ritem Developed web application for managing large scale deployments of User Behavior sensors and data loss detection using Java Spring Framework with AngularJS front end and Hibernate with Postgres and Mongodb databases. Designed and implement REST API using Swagger specification and Oauth2 authentication.
%\ritem Developed desktop client application for large scale document tracker injection using Electron and AngularJS.
%\ritem Automated deployment and testing of new builds using Beanstalk and AWS Elastic Computing resources.
%\ritem Integrated high speed file system backed by MongoDb and optimized process for tracking new documents, increasing speed by a factor of over 270\%.
%\ritem Designed and implemented new method for tracking document usage in the cloud using Google Drive API.
\ritem Developed midstream network based document interception and tracking system using Squid Proxy and ICAP.
    %server to automatically inject trackers into suspicious documents.
\end{ritemize}

\rjob{\textbf{Robotics Software Consultant}, 3DDataLtd}{May 2015 - Aug 2015}\\
\begin{ritemize}
\ritem Developed a framework for 3d Simultaneous Localization and Mapping using the Lidar Odometry and Mapping algorthim. Modified algorithm to integrate IMU Sensor data using an Extended Kalman Filter.
    %Implemented in C++ with PCL and ROS.
\end{ritemize}

\rjob{\textbf{Radar Software Engineer}, Raytheon}{Sep 2013 - April 2015}\\
\begin{ritemize}
%\ritem Maintained radar configuration and verification tools in Perl and Cshell.
\ritem Analyzed data from radar testing and provide software support for automated calibration, satellite tracking, antenna diagnostics, and maintenance prioritization software in Ada and C++.
%\ritem Developed additions to radar software in Ada and C++ for calibration and diagnostics.
\ritem Earned team achievement award for completing new diagnostic capabilities ahead of schedule.
\end{ritemize}



%\clearpage{}



%\rjob{\textbf{Writing Associate}, Swarthmore College}{Sep 2011- Dec 2012}\\
%\begin{ritemize}
%\ritem Selected for excellent written and oral communication skills to work in Swarthmore writing program.
%\ritem Conducted half hour conferences with students for revision and good writing practice in college writing center.
%\ritem Assisted one professor per semester in teaching a writing intensive course, meeting regularly with 10-14 students to work on writing skills and assist in planning and revising papers.
%\end{ritemize}


%\rjob{\textbf{IT Associate}, Swarthmore College}{Fall 2011}\\
%\begin{ritemize}
%\ritem Selected for excellent communication skills and practical problem solving ability to work on college IT team.
%\ritem Fielded calls and walk-ins at IT support office. Addressed security and networking issues on PCs.
%\end{ritemize}



%\rjob{\textbf{Intern}, Verizon Laboratories}{January 2011}\\
%\begin{ritemize}
%\ritem Shadowed Verizon Interface Design Engineer developing Videochat interface for smartphones.
%\ritem Evaluated usability of current videochat apps and prepared recommendations for videochat interface design.
%\end{ritemize}

%%%%%%%%%%%%%%%%%%%%%%%%% Extracurricular %%%%%%%%%%%%%

\vspace{0.2cm}
\rsection{Publications:}

\hspace*{-0.1cm}
\begin{minipage}{1.01\linewidth}
\begin{itemize}[label={},itemindent=-2em,leftmargin=2em, parsep=5pt]
  \item {\bf [Oakland S\&P 2023]}
  \textit{Precise Detection of Kernel Data Races with Probabilistic Lockset Analysis.}
    G. Ryan, A. Shah, D. She, S. Jana. \url{https://cs.columbia.edu/~gabe/files/oakland2023_pla.pdf}

  \item {\bf [ICSE 2022]}
    %\textrm{\fontsize{11pt}{12pt}\selectfont TOGA: A Neural Method for Test Oracle Generation.}
    \textit{TOGA: A Neural Method for Test Oracle Generation.}
    E. Dinella*, G. Ryan*, T. Mytkowitz, S. Lahiri. \url{https://arxiv.org/pdf/2109.09262.pdf} \textbf{(Distinguished Paper Award)}
  \item {\bf [OSDI 2021]}
    \textit{DistAI: Data-Driven Automated Invariant Learning for Distributed Protocols.}
    J. Yao, R. Tao, R. Gu, J. Nieh, S. Jana, G. Ryan. \url{https://www.usenix.org/system/files/osdi21-yao.pdf} \textbf{(Best Paper Award)}
  \item {\bf [USENIX Security 2021]}
    \textit{Fine Grained Dataflow Tracking with Proximal Gradients.}
    G. Ryan, A. Shah, D. She, K. Bhat, and S. Jana. \url{https://arxiv.org/abs/1909.03461}
  \item {\bf [PLDI 2020]}
    \textit{Learning Nonlinear Loop Invariants with Gated Continuous Logic Networks.}
    J. Yao*, G. Ryan*, J. Wong*, S. Jana, and R. Gu. \url{https://arxiv.org/abs/2003.07959}
  \item {\bf [ICLR 2020]}
    \textit{CLN2INV: Learning Loop Invariants with Continuous Logic Networks.}
    G. Ryan*, J. Wong*, Y. Jianan*, R. Gu, and S. Jana. \url{https://arxiv.org/abs/1909.11542}
  \item {\bf [InfoVis 2018]}
    \textit{At a Glance: Pixel Approximate Entropy as a Measure of Line Chart Complexity.}
    G. Ryan, A. Mosca, R. Chang, and E. Wu. \url{https://arxiv.org/abs/1811.03180}
  \item  {\bf [Oakland S\&P Workshops 2018]}
    \textit{Simulated User Bots: Real Time Testing of Insider Threat Detection Systems.}
    P. Dutta, G. Ryan, A. Zieba, and S. Stolfo. \url{https://ieeexplore.ieee.org/document/8424654}
  \item  {\bf [Oceans 2012]} \textrm{\fontsize{11pt}{12pt}\selectfont Oversampling MAVS for reduction of vortex-shedding velocity-sensing noise.} A. Williams, G. Ryan, and T. Thwaites. \url{https://ieeexplore.ieee.org/document/6404777}
\end{itemize}
\end{minipage}


\vspace{0.2cm}
\rsection{Awards:}
\hspace*{-0.1cm}

\vspace{0.2cm}
\begin{minipage}{1.01\linewidth}
\begin{itemize}[label={},itemindent=-2em,leftmargin=2em, parsep=4pt]
  \item {\bf National Defense Science and Engineering Graduate Fellowship} (NDSEG). Awarded NDSEG Fellowship for proposal "Proximal Gradient Analysis for Vulnerability Detection and Defense." 2019.
  \item {\bf NSF Graduate Research Fellowship Honorable Mention}. Accorded honorable mention for proposal "Modeling and Simulating Adversarial User Behavior with Sequential VAEs." 2018.
  \end{itemize}
\end{minipage}

\pagebreak{}

\rsection{Academic Service:}
\hspace*{-0.1cm}
\begin{minipage}{1.01\linewidth}
\begin{itemize}[label={},itemindent=-2em,leftmargin=2em, parsep=4pt]
  \item AAAI 2023. Program Committee.
  \item TOSEM 2022. Reviewer.
  \item AAAI 2022. Reviewer.
  \end{itemize}
\end{minipage}

\vspace{0.2cm}
\rsection{Teaching:}
\hspace*{-0.1cm}
\begin{minipage}{1.01\linewidth}
\begin{itemize}[label={},itemindent=-2em,leftmargin=2em, parsep=4pt]
  \item Teaching Assistant. Intrusion Detection Systems. Fall 2022.
  \item Lecturer. Continuous Logic Networks. Spring 2021.
  \item Teaching Assistant. Intrusion Detection Systems. Fall 2020.
  \end{itemize}
\end{minipage}

%%%%%%%%%%%%%%%%%%%%%%%%% Technical Skills %%%%%%%%%%%%%%%%%%%%%%%%%
\vspace{0.2cm}
\rsection{Technical Summary:}

\hspace*{-0.3cm}
\textbf{Languages:} Proficient in Python, C/C++, Java, Javascript, and SQL. \\
\hspace*{-0.3cm}
\textbf{Skills:} Linux, Cloud Compute, Transformers, Pytorch, Tensorflow, LLVM, Pandas, Z3



\end{document}
